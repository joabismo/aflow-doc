%%%%%%%%%%%%%%%%%%%%%%%%%%%%%%%%%%%%%%%%%
% Beamer Presentation
% LaTeX Template
% Version 1.0 (10/11/12)
%
% This template has been downloaded from:
% http://www.LaTeXTemplates.com
%
% License:
% CC BY-NC-SA 3.0 (http://creativecommons.org/licenses/by-nc-sa/3.0/)
%
%%%%%%%%%%%%%%%%%%%%%%%%%%%%%%%%%%%%%%%%%
%----------------------------------------------------------------------------------------
%	PACKAGES AND THEMES
%----------------------------------------------------------------------------------------

\documentclass{beamer}

\mode<presentation> {

% The Beamer class comes with a number of default slide themes
% which change the colors and layouts of slides. Below this is a list
% of all the themes, uncomment each in turn to see what they look like.

%\usetheme{default}
%\usetheme{AnnArbor}
%\usetheme{Antibes}
%\usetheme{Bergen}
%\usetheme{Berkeley}
%\usetheme{Berlin}
%\usetheme{Boadilla}
%\usetheme{CambridgeUS} %Color rojo, me gusta
%\usetheme{Copenhagen} %Ta bueno
%\usetheme{Darmstadt} % Este es el 1er candidato
%\usetheme{Dresden} % Es lindo y parecido al de arriba
\usetheme{Frankfurt}
%\usetheme{Goettingen}
%\usetheme{Hannover}
%\usetheme{Ilmenau}
%\usetheme{JuanLesPins}
%\usetheme{Luebeck}
%\usetheme{Madrid}
%\usetheme{Malmoe}
%\usetheme{Marburg}
%\usetheme{Montpellier}
%\usetheme{PaloAlto}
%\usetheme{Pittsburgh}
%\usetheme{Rochester}
%\usetheme{Singapore}
%\usetheme{Szeged}
%\usetheme{Warsaw}

% As well as themes, the Beamer class has a number of color themes
% for any slide theme. Uncomment each of these in turn to see how it
% changes the colors of your current slide theme.

%\usecolortheme{albatross}
%\usecolortheme{beaver}
%\usecolortheme{beetle}
%\usecolortheme{crane}
%\usecolortheme{dolphin}
%\usecolortheme{dove}
%\usecolortheme{fly}
%\usecolortheme{lily}
%\usecolortheme{orchid}
%\usecolortheme{rose}
%\usecolortheme{seagull}
%\usecolortheme{seahorse}
%\usecolortheme{whale}
%\usecolortheme{wolverine}

%\setbeamertemplate{footline} % To remove the footer line in all slides uncomment this line
%\setbeamertemplate{footline}[page number] % To replace the footer line in all slides with a simple slide count uncomment this line

%\setbeamertemplate{navigation symbols}{} % To remove the navigation symbols from the bottom of all slides uncomment this line
}

\usepackage{graphicx} % Allows including images
\usepackage{booktabs} % Allows the use of \toprule, \midrule and \bottomrule in tables
\usepackage{listings}
%\usepackage{url}
\usepackage{hyperref}
\usepackage[spanish]{babel}
\usepackage[utf8]{inputenc}
\usepackage{listings}
\lstset{%                       % Configuración de parámetros de listing.
  basicstyle=\small\ttfamily,     % Códigos con fuente TrueType.
  breaklines=true,                % Rompe líneas demasiado largas.
  xrightmargin=1cm,               % Margen derecho.
%  escapeinside=wz,                % Para escapar a LaTeX.
}%

%----------------------------------------------------------------------------------------
%	TITLE PAGE
%----------------------------------------------------------------------------------------

\title[Flujo Analógico]{Edición de Circuitos en Electric VLSI y simulación en gnucap} % The short title appears at the bottom of every slide, the full title is only on the title page

\author{Leandro Marsó} % Your name
\institute[] % Your institution as it will appear on the bottom of every slide, may be shorthand to save space
{
Córdoba\\ % Your institution for the title page
\medskip
\textit{elleandro@gmail.com} % Your email address
}
\date{\today} % Date, can be changed to a custom date

\begin{document}


\lstset{
basicstyle=\ttfamily,                   % Code font, Examples: \footnotesize, \ttfamily
frame=none,                             % A frame around the code
tabsize=2,                              % Default tab size
captionpos=b,                           % Caption-position = bottom
breaklines=false,                        % Automatic line breaking?
breakatwhitespace=false,                % Automatic breaks only at whitespace?
showspaces=false,                       % Dont make spaces visible
showtabs=false,                         % Dont make tabls visible
}

\begin{frame}
\titlepage % Print the title page as the first slide
\end{frame}

\begin{frame}
\frametitle{Contenido} % Table of contents slide, comment this block out to remove it
\tableofcontents % Throughout your presentation, if you choose to use \section{} and \subsection{} commands, these will automatically be printed on this slide as an overview of your presentation
\end{frame}

%----------------------------------------------------------------------------------------
%	PRESENTATION SLIDES
%----------------------------------------------------------------------------------------
\section{Edición de Circuitos} % Sections can be created in order to organize your presentation into discrete blocks, all sections and subsections are automatically printed in the table of contents as an overview of the talk

%------------------------------------------------
\begin{frame}
\subsection{Circuitos Esquemáticos}
\frametitle{Edición de Circuitos Esquemáticos}
\end{frame}
%------------------------------------------------
\begin{frame}
\subsubsection{Crear una librería}
\frametitle{Creación de una librería}
Creamos la librería que contendrá las celdas que vamos a diseñar:

File $\rightarrow$ New library

Y elegimos el nombre \textbf{analog}
\begin{figure}
\includegraphics[width=0.35\linewidth]{figuras/edicionElectric.png}
\end{figure}
\end{frame}
%------------------------------------------------
\begin{frame}
\subsubsection{Crear una nueva celda }
\frametitle{Nueva Celda}
Creamos el esquemático de la nueva celda, con el nombre \textbf{nmosMin}: 

Cell $\rightarrow$ New Cell.
Y en View: elegimos \textbf{schematic}
\begin{figure}
\includegraphics[width=0.45\linewidth]{figuras/edicionElectric-1.png}
\end{figure}
\end{frame}
%------------------------------------------------
\begin{frame}
\subsubsection{Instanciar un transistor}
\frametitle{Instanciar un transistor}
Vamos a la pestaña Components y seleccionamos un transistor nmos de cuatro terminales y lo instanciamos en el plano de trabajo:
\begin{figure}
\includegraphics[width=0.5\linewidth]{figuras/edicionElectric-2b.png}
\end{figure}
\end{frame}
%------------------------------------------------
\begin{frame}
\frametitle{Editar propiedades}
Para cambiar las propiedades del transistor hacemos \textbf{Ctrl-I} y cambiamos el nombre del transistor, y dejamos el ancho y largo en 2:
\begin{figure}
\includegraphics[width=0.5\linewidth]{figuras/edicionElectric-3.png}
\end{figure}
\end{frame}
%------------------------------------------------
\begin{frame}
\frametitle{Conectar y crear puertos}
Ahora vamos a crear puertos y exportarlos para ejemplificar el uso del simulador y una convención para los nombres.
\vspace{0.5cm}
\begin{columns}[t]
\column{.4\textwidth}
Conectamos el drenador haciendo click izquierdo sobre el mismo y luego hacemos click derecho sobre otro lado para conectar:
\begin{figure}
\includegraphics[width=0.45\linewidth]{figuras/edicionElectric-4.png}
\end{figure}
\column{.4\textwidth}
Nombramos el cable haciendo click izquierdo exactamente en el centro del mismo y luego \textbf{Ctrl-I} para cambiarle el nombre:
\begin{figure}
\includegraphics[width=0.45\linewidth]{figuras/edicionElectric-4b.png}
\end{figure}
\end{columns}
\end{frame}
%------------------------------------------------
\begin{frame}
\frametitle{Conectar y crear puertos}
%Ahora vamos a crear puertos y exportarlos para ejemplificar el uso del simulador y una convención para los nombres.
%\vspace{0.5cm}
\begin{columns}[t]
\column{.45\textwidth}
Nombramos todos los puertos del transistor de la siguiente forma:
\begin{figure}
  \includegraphics[width=0.75\linewidth]{figuras/edicionElectric-4c.png}\label{fig:nmosMin}
\end{figure}
\column{.4\textwidth}
Ahora vamos a crear los puertos de esta celda, instanciamos un elemento llamado \textbf{off-page} como muestra la figura:
\begin{figure}
\includegraphics[width=1.25\linewidth]{figuras/edicionElectric-4d.png}
\end{figure}
\end{columns}
\end{frame}
%------------------------------------------------
\begin{frame}
\frametitle{Conectar y crear puertos}
\begin{columns}[t]
\column{.45\textwidth}
Creamos el puerto haciendo un click izquierdo sobre el elemento \textbf{off-page} y presionamos \textbf{Ctrl-E} para exportar este puerto y darle un nombre. Además conectamos un cable y le damos el mismo nombre que el del drenador del transistor:
\begin{figure}
  \includegraphics[width=0.90\linewidth]{figuras/edicionElectric-4e.png}
\end{figure}
\column{.4\textwidth}
Lo mismo hacemos para los otros puertos, como mostramos en la siguiente figura:
\begin{figure}
\includegraphics[width=1.25\linewidth]{figuras/edicionElectric-4f.png}
\end{figure}
\end{columns}
\end{frame}

%------------------------------------------------

\begin{frame}
\frametitle{Detectar errores de conexión}
Si presionamos \textbf{F5} durante la edición, nos realizará un chequeo de errores, que pueden ser detallados presionando \textbf{Shift $+<$} o \textbf{Shift $+>$}. Damos dos ejemplos típicos:
\begin{columns}[t]
\column{.45\textwidth}
\textbf{Cables desconectados}
\begin{figure}
  \includegraphics[width=1.0\linewidth]{figuras/edicionElectric-4g.png}
\end{figure}
\scriptsize{Este error aparece cuando un cable no tiene conexión o nombre.} 
\column{.45\textwidth}
\textbf{Pin innecesario}
\framezoom<1><2>[border](2.59cm,2.95cm)(1.5cm,1.5cm)
\framezoom<1><3>[border](8.09cm,2.75cm)(1.5cm,1.5cm)
\begin{figure}
\includegraphics[width=1.0\linewidth]{figuras/edicionElectric-4h.png} 
\end{figure}
\scriptsize{Este error se produce cuando hacemos un cable sobre la misma línea recta en dos tramos, es decir, presionando dos veces el click derecho para realizar un cable recto.} 
\end{columns}
\end{frame}

%------------------------------------------------
\begin{frame}
\frametitle{Corrección de errores}
Para corregir el error de pines extra:

\textbf{Edit $\rightarrow$ Cleanup cell $\rightarrow$ Cleanup Pin}

\end{frame}

%------------------------------------------------
\begin{frame}
\frametitle{Crear un ícono}
Luego de finalizar con el diseño esquemático, queremos tener un ícono para ser instanciado en otros esquemáticos:


\textbf{View $\rightarrow$ Make Icon View}

\begin{figure}
\includegraphics[width=0.8\linewidth]{figuras/edicionElectric-5.png} 
\end{figure}

\end{frame}
%%----------------------------------------------
\begin{frame}
\frametitle{Editar el ícono}
Para editar este ícono hacemos:
\textbf{View $\rightarrow$ Edit Icon View}

Seleccionamos el rectángulo y con \textbf{Ctrl-i} le ajustamos el valor de X para que sea cuadrado. Además movemos los puertos del ícono para separarlos y centrarlos:

\begin{figure}
\includegraphics[width=0.99\linewidth]{figuras/edicionElectric-5b.png} 
\end{figure}

\end{frame}
%--------------------------------------------------
\begin{frame}
\subsection{Layout}
\frametitle{Creación del layout}
Para realizar la versión en \textit{\textbf{layout}} de esta celda, hacemos:

\textbf{View $\rightarrow$ Edit Layout View}
\begin{figure}
  \includegraphics[width=0.50\linewidth]{figuras/edicionElectric-6.png}
\end{figure}
Y elegimos crear una celda vacía de layout.
\end{frame}
%------------------------------------------------
\begin{frame}
\frametitle{Creación del layout - Instanciar elementos}
\begin{columns}[c]
\column{.4\textwidth}
Vamos a instanciar un transistor tipo n (\textbf{nMos}), los contactos a Metal1 (\textbf{nAct}) para este transistor, y el contacto a bulk necesario (\textbf{pWell}). Esos elementos los encontramos en la pestaña de \textbf{Components}, como lo señalamos en la figura:
\column{.4\textwidth}
\begin{figure}
  \includegraphics[width=0.69\linewidth]{figuras/edicionElectric-6bbb.png}
\end{figure}
\end{columns}
\end{frame}
%------------------------------------------------
\begin{frame}
\frametitle{Creación del layout - Instanciar elementos}
\begin{figure}
  \includegraphics[width=1.00\linewidth]{figuras/edicionElectric-7.png}
\end{figure}
\end{frame}
%------------------------------------------------
%\begin{frame}
%\frametitle{Creación del layout - Instanciar elementos}
%\begin{columns}[c]
%\column{.4\textwidth}
%Vamos a instanciar un transistor tipo n (nMos), los contactos a M1 para este transistor, y el contacto a bulk necesario (pWell). Esos elementos los encontramos en la pestaña de \textbf{Components}, como lo señalamos en la figura:
%\column{.4\textwidth}
%\begin{figure}
%  \includegraphics[width=0.69\linewidth]{figuras/edicionElectric-6bbb.png}
%\end{figure}
%\end{columns}
%\end{frame}
%%------------------------------------------------
\begin{frame}
\frametitle{Creación del layout - Contactos a Metal 1}
Haciendo click izquierdo al contacto y arrastrando hacia la derecha para acercarlo al transistor, vemos una leyenda que nos dá información sobre la regla de DRC.
\begin{figure}
  \includegraphics[width=0.89\linewidth]{figuras/edicionElectric-8.png}
\end{figure}
%\scriptsize{
%}

\end{frame}
%------------------------------------------------
\begin{frame}
\frametitle{Creación del layout - Contactos a Metal 1}
%\scriptsize{
Lo acercamos tanto como se pueda, hasta que nos indique que no cumple una regla de DRC:
%}
\begin{figure}
  \includegraphics[width=0.89\linewidth]{figuras/edicionElectric-8bb.png}
\end{figure}
\end{frame}
%------------------------------------------------
\begin{frame}
\frametitle{Creación del layout - Contactos a Metal 1}
%\scriptsize{
Asi queda bien conectado:
%}
\begin{figure}
  \includegraphics[width=0.89\linewidth]{figuras/edicionElectric-8bbb.png}
\end{figure}
\end{frame}
%------------------------------------------------
\begin{frame}
\frametitle{Creación del layout - Contactos a Metal 1}
%\scriptsize{
Hacemos lo mismo con el otro terminal:
%}
\begin{figure}
  \includegraphics[width=0.89\linewidth]{figuras/edicionElectric-8bbbb.png}
\end{figure}
\end{frame}
%------------------------------------------------
\begin{frame}
\frametitle{Creación del layout - Contactos a bulk}
%\scriptsize{
De igual forma conectamos el bulk\footnote{Modificamos el valor de X al contacto a bulk para que sea más grande} al source (siempre presionando \textbf{F5} para chequear DRC):
%}
\begin{columns}
  \column{.54\textwidth}
\begin{figure}
  \includegraphics[width=0.89\linewidth]{figuras/edicionElectric-9.png}
\end{figure}
\column{.54\textwidth}
\begin{figure}
  \includegraphics[width=0.89\linewidth]{figuras/edicionElectric-9a.png}
\end{figure}

\end{columns}

\end{frame}
%-------------------------------------------------------------------------------
\begin{frame}{Conexión simbólica del \emph{body} del transistor}
  Seleccionar \emph{Toggle Special Select}, hacer click izquierdo sobre el transistor y luego click derecho sobre los contactos a bulk\footnote{Para mas explicaciones ver \textbf{Body Checking Section} (Capítulo 9 del manual de \href{http://staticfreesoft.com/jmanual/ElectricManual-9.05.pdf}{Electric})}.
\begin{figure}
\includegraphics[width=0.51\linewidth]{figuras/edicionElectric-9b.png}
\end{figure}
\end{frame}

%%%------------------------------------------------
\begin{frame}
\frametitle{Creación del layout - Contacto a poly}
Instanciamos un contacto a poly y lo conectamos al gate: 
\begin{figure}
  \includegraphics[width=0.89\linewidth]{figuras/edicionElectric-10.png}
\end{figure}
\end{frame}
%------------------------------------------------
\begin{frame}
\frametitle{Creación del layout - Puertos}
Creamos los puertos con los mismos nombres que en el esquemático, seleccionamos primero el lugar donde queremos crear el puerto, y luego presionamos \textbf{Ctrl-e}:
\begin{figure}
  \includegraphics[width=0.59\linewidth]{figuras/edicionElectric-11.png}
\end{figure}
\end{frame}
%------------------------------------------------
\begin{frame}
\frametitle{Creación del layout - Puertos}
De la misma forma creamos el puerto para \textbf{Vd} y \textbf{AVSS}:
\begin{figure}
  \includegraphics[width=0.51\linewidth]{figuras/edicionElectric-11a.png}
\end{figure}
\end{frame}
%------------------------------------------------
\subsection{DRC}
\begin{frame}
\frametitle{Comprobación de errores de DRC}
Durante la edición del \emph{layout}, es necesario asegurarnos que no haya violacioes a las reglas de diseño, para eso presionamos \textbf{F5} y correjimos los errores a cada paso.
\end{frame}
%------------------------------------------------
\subsection{LVS}
\begin{frame}{Chequéo errores de LVS}
  \textbf{Electric} cuenta con una herramienta llamada \textbf{NCC} (\emph{Network Consistency Check} que nos permite realizar una verificación de equivalencia entre el esquemático y el \emph{layout} de la misma celda. También sirve para hacer equivalencia entre esquemáticos, entre \emph{layouts}, y otras vistas de la misma celda.
\end{frame}
%------------------------------------------------
\begin{frame}
\frametitle{Verificación de equivalencia entre el esquemático y el layout (LVS)}
\scriptsize{Primero configuramos correctamente la herramienta para que el LVS se haga correctamente. Vamos a \textbf{File $\rightarrow$ Preferences $\rightarrow$ NCC} y nos aseguramos seleccionar donde dice \textbf{Check transistor sizes} y en \textbf{Check transistor body connections}, como mostramos en la figura:
}
\begin{figure}
  \includegraphics[width=0.75\linewidth]{figuras/edicionElectric-12.png}
\end{figure}
\end{frame}
%------------------------------------------------
\begin{frame}{Realizar el LVS}
  \begin{figure}
    \includegraphics[width=1.10\linewidth]{figuras/edicionElectric-9c.png}
  \end{figure}
\end{frame}
%-----------------------------------------------------------------------
\begin{frame}[fragile]{Mensaje de LVS correcto}
  Si el esquemático y el \emph{layout} son equivalentes, el mensaje será:
  \begin{tiny}
  \begin{verbatim}
  =================================6=================================
  Hierarchical NCC every cell in the design: cell 'nmosMin{sch}'  cell 'nmosMin{lay}'
  Comparing: analog:nmosMin{sch} with: analog:nmosMin{lay}
    exports match, topologies match, sizes match in 0.004 seconds.
    Summary for all cells: exports match, topologies match, sizes match
    NCC command completed in: 0.006 seconds.
  \end{verbatim}
\end{tiny}
\end{frame}
%-----------------------------------------------------------------------
\subsection{Extracción de los parásitos}
\begin{frame}[fragile]
\frametitle{Extracción de los parásitos del layout}
Realizamos ahora una extracción del circuito representado por el \emph{layout} para crear un \emph{netlist} SPICE, que incluya los elementos parásitos representados por los contactos, vías y metales utilizados para realizar las interconexiones.

\begin{exampleblock}{Extracción de parásitos}
\textbf{
Tools $\rightarrow$ Simulation (Spice) $\rightarrow$ Write Spice Deck
}
\end{exampleblock}

Luego especificamos donde guardar el archivo, y damos nombre al archivo. Creamos una carpeta llamada \verb.sim. y le ponemos el nombre \verb(nmosMin_lay.spi(. Luego examinar el archivo y ver el circuito y las resistores y capacitores que aparecen.

\end{frame}
%------------------------------------------------
\section{Simulaciones}
%\subsection{Esquemático y Layout}
\begin{frame}[fragile]
  \frametitle{Simulación del \emph{Layout}}
  Creamos una nueva celda de \emph{layout} llamada \verb.nmosMin_tb. donde instanciamos nuestro transistor.
  \begin{exampleblock}{Instanciar una celda de \emph{layout}}
\textbf{
Components $\rightarrow$ Cell $\rightarrow$ nmosMin\{lay\}} (y ubicarla en el plano)
  \end{exampleblock}

  Exportamos los tres puertos con los nombre Vg, Vd y vss, y luego creamos el netlist para simulación:
   \begin{exampleblock}{Creación del netlist para simulación}
   \textbf{
Tools $\rightarrow$ Simulation (Spice) $\rightarrow$ Write Spice Deck
}
   \end{exampleblock}

\end{frame}
%------------------------------------------------
\begin{frame}[fragile]
  \frametitle{Simulación del \emph{Layout} - Creación de un testbench}
  Creamos el archivo \verb_simulacion.gnucap_ con el siguiente contenido:
  \begin{tiny}
%    \lstset{breaklines=true,extendedchars=true,mathescape=false} 
%    \lstinputlisting{gnucap/simulacion.gnucap}
\end{tiny}
\href{run:gnucap/simulacion.gnucap}{Descargar el testbench}. Descargar el archivo \href{run:gnucap/fuentes.spi}{fuentes.spi}. 
\end{frame}
%----------------------------------------------------
\begin{frame}{Análisis DC}
Para cualquier tipo de simulación, debo especificar si quiero ver o guardar algunas señales. Por ejemplo, recién hemos utilizado:
  \begin{exampleblock}{Selecciono que señales mostrar}
    print dc V(Vd) V(Vg) I(Vdd)
  \end{exampleblock}

Para lanzar la simulación haciendo un análisis en DC:

  \begin{exampleblock}{Comando de simulación}
dc Vdd 0 3.3 0.05 Vin 0.00 3.3 0.66 
  \end{exampleblock}

  Documentación completa de este análisis y las opciones:
  \begin{tiny}
  \url{http://www.gnucap.org/dokuwiki/doku.php?id=gnucap:manual:commands:dc}
  \end{tiny}

\end{frame}
%-------------------------------------------------------------------------------
\begin{frame}[fragile]{Lanzar la simulación}
  Para comenzar la simulación, desde una consola en el directorio \verb.sim. hacemos:
  \begin{verbatim}
  $gnucap -i simulacion.gnucap
  \end{verbatim}

  Al finalizar la simulación, \textbf{gnucap} nos devolverá el control y podremos continuar con mas simulaciones, o salir del programa presionando \textbf{Ctrl-d}. 
\end{frame}
%----------------------------------------------------
\begin{frame}[fragile]{Visualización de la simulación}
  Para visualizar los datos de la simulación que guardamos en el archivo simulacion.out optamos por utilizar un script en Python:
  \begin{verbatim}
  $python/plot_transistor_curves.py sim/simulacion.out
  \end{verbatim}

  El resultado de esta simulación se puede ver en la filmina siguiente.

\end{frame}
 
%------------------------------------------------
\subsection{Caracterización de los transistores}
\begin{frame}
  \frametitle{Caracterización de los transistores de canal N y P por medio de simulación (familia de curvas de Id/Vds)}
  \begin{figure}
  \includegraphics[width=0.95\linewidth]{figuras/nmosMin-IDS-VDS.pdf}
  \end{figure}

\end{frame}
%----------------------------------------------------
\begin{frame}
\frametitle{Simulación del esquemático}
  La simulación del circuito esquemático se realiza de la misma forma que el \emph{layout}, sólo que se debe crear el netlist Spice desde la vista de esquemático.

\end{frame}
%------------------------------------------------
\begin{frame}
\frametitle{Simulaciones de punto de operación}
Para este comando, también es necesario determinar las señales que queremos ver, lo hacemos con el comando \textbf{print}

  \begin{exampleblock}{Comando para agregar valores que se mostrarán}
    * Tensión en todos los nodos del circuito:

    \textbf{print \alert{op} v(nodes)}

    * Agrega a la lista la corriente que da la fuente:

  \textbf{print \alert{op} +i(Vdd)}  
\end{exampleblock}
\end{frame}

%------------------------------------------------
\begin{frame}
  \frametitle{Más sobre el comando Print}
%\textbf{Ver elementos internos de los subcircuitos}

Por ejemplo, cuando está instanciado el circuito lm\_2stage\_opamp, con nombre Xlm\_2stag\_0:
\begin{exampleblock}{Ver si un transistor está saturado}

 * Instanciación del circuito:

 * Xlm\_2stag\_0 ( vss i\_bias vdd out )  lm\_2stage\_opamp 

 * Para acceder a un elemento dentro del circuito, por ejemplo el transistor MM1:

 * MM1.Xlm\_2stag\_0

 * Por ejemplo quiero ver si está saturado ese transistor:

  print op saturated(MM1.Xlm\_2stag\_0)
\end{exampleblock}
 Más info sobre como llamar a los elementos internos de los circuitos en:
  \begin{footnotesize}
   \url{http://gnucap.org/gnucap-man-html/gnucap-man083.html}
  \end{footnotesize}

\end{frame}
%------------------------------------------------------------------------
\begin{frame}[fragile]
  \frametitle{Simulaciones de punto de operación}

 Sintáxis:
 \begin{verbatim}
 op start stop stepsize {options ...} 
 \end{verbatim}
 \begin{exampleblock}{Algunos ejemplos}
   * Hacer una simulación a 27 grados:

   op 27

   * Hacer un barrido de temperatura desde -40 a 180 en pasos de 10 grados, haciendo una simulación de OP en cada paso:

   op -40 180 10
 \end{exampleblock}
 Existen muchas opciones, por ejemplo que el barrido sea logarítmico, inverso, cíclico, etc. Para mas opciones ver:
  \begin{tiny}
    \url{http://www.gnucap.org/dokuwiki/doku.php?id=gnucap:manual:commands:op}
  \end{tiny}

 


\end{frame}
%------------------------------------------------
\subsection{Régimen transitorio}
\begin{frame}
\frametitle{Régimen transitorio}

\end{frame}
%------------------------------------------------
\subsection{Análisis AC}
\begin{frame}
\frametitle{Análisis AC}
\end{frame}
%------------------------------------------------
\subsection{Transformada de Fourier}
\begin{frame}
\frametitle{Transformada de Fourier}
\end{frame}
%------------------------------------------------
%\subsection{Alternativas de Interacción con circuitos digitales}
%\begin{frame}
%\frametitle{Alternativas de Interacción con circuitos digitales}
%\end{frame}
%------------------------------------------------

\begin{frame}
\Huge{\centerline{Fin}}
\end{frame}

%----------------------------------------------------------------------------------------

\end{document} 
